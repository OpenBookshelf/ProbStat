\documentclass[12pt]{article}
\usepackage{amsmath}
%\usepackage{xcolor}
\usepackage{xepersian}
\settextfont{B Nazanin}
\linespread{1.4}

\begin{document}
	\subsection*{مقدمه‌ای بر نظریه مجموعه‌ها}
	\textbf{مجموعه:}
	دسته‌ای از عناصر متمایز را مجموعه گویند.\\
	\textbf{زیرمجموعه:} 
	اگر هر عضو مجموعه $A$ متعلق به مجموعه $B$ نیز باشد، مجموعه $A$ زیرمجموعه مجموعه $B$ است. به بیان ریاضی: $A\subset B$\\
	\textbf{مجموعه مساوی:} 
	اگر و تنها اگر $A\subset B$ و $B\subset A$ آنگاه دو مجموعه $A$ و $B$ مساوی یکدیگر هستند. \\
	\textbf{مجموعه مرجع:} 
	مجموعه‌ای که شامل تمام عناصر است و با $\Omega$ نمایش داده می‌شود. \\
	\textbf{مجموعه تهی:} 
	مجموعه‌ای که فاقد عضو است و با $\{\}$ یا $\emptyset$ نمایش داده می‌شود. \\
	\textbf{اجتماع دو مجموعه:}
	مجموعه عناصری که یا در $A$ یا در $B$ باشند. به بیان ریاضی: $A\cup B$\\
	\textbf{اشتراک دو مجموعه:}
	مجموعه عناصری که هم در $A$ و هم در $B$ باشند. به بیان ریاضی: $A\cap B$\\
	\textbf{دو مجموعه جدا از هم:}
	دو مجموعه که هیچ عضو مشترکی ندارند. به بیان ریاضی: $A\cap B = \emptyset$\\
	\textbf{مکمل یک مجموعه:}
	تمام اعضای مجموعه مرجع که در مجموعه $A$ نباشند را مجموعه مکمل $A$ می‌نامند و با $A^c$ یا $\bar{A}$ نمایش می‌دهند. \\
	\textbf{تفاضل دو مجموعه:}
	عناصری که تنها در یکی از دو مجموعه $A$ و $B$ باشند. به بیان ریاضی:\\ 
	$A-B=A\cap \bar{B}$ یا $B-A=B\cap \bar{A}$
	\subsection*{قوانین پایه نظریه مجموعه‌ها}
	\begin{enumerate}
		\item $A\cup \Omega=\Omega$
		\item $A\cap \Omega=A$
		\item $A\cup \emptyset=A$
		\item $A\cap \emptyset=\emptyset$
		\item $B\subset A \Rightarrow A\cup B=B$
		\item $B\subset A \Rightarrow A\cap B=A$
		\item $A\subset B \; ,B\subset C \Rightarrow A\subset C$
		\item $A\cup B=B\cup A \;, A\cap B=B\cap A$
		\item $A\cup (B\cup C)=(A\cup B)\cup C \;, A\cap (B\cap C)=(A\cap B)\cap C$
		\item $A\cup (B\cap C)=(A\cup B)\cap (A\cup C)$
		\item $A\cap (B\cup C)=(A\cap B)\cup (A\cap C)$
		\item $\overline{(A\cup B)}=\bar{A}\cap \bar{B} \;, \overline{(A\cap B)}=\bar{A}\cup \bar{B}$
		
	\end{enumerate}
	
	\subsection*{افراز}
	مجموعه‌های غیرتهی $A_1,A_2,...,A_m$ را افرازی از مجموعه مرجع $\Omega$ می‌نامیم چنانچه: 
	\begin{align*}
		\forall i \neq j;\: A_i \cap A_j=\emptyset\\
		A_1 \cup A_2 \cup ...\cup A_m=\Omega
	\end{align*}	
	\subsection*{حاصلضرب دکارتی} 
	حاصلضرب دکارتی دو مجموعه $A$ و $B$ (مجموعه $A$ با عناصر $\alpha_i$ و مجموعه $B$ با عناصر $\beta_j$) عبارت است از مجموعه تمام زوج مرتب‌های $(\alpha_i,\beta_j)$ و به صورت $A\times B=C$ نمایش داده می‌شود به طوری‌که اگر $A$ دارای $m$ عضو و $B$ دارای $n$ عضو باشد، مجموعه $C$ دارای $mn$ عضو خواهد بود.
	
	\subsection*{فضای احتمال: $(\Omega,F,P)$}
	فضای احتمال یا مدل احتمالاتی یک آزمایش تصادفی از عوامل زیر تشکیل می‌شود:\\
	\textbf{مجموعه$\Omega$:}
	تمام نتایج ممکن $\omega_i$ برای آزمایش را شامل می‌شود.\\ 
	\textbf{مجموعه $F$:}
	زیرمجموعه‌های مجموعه $\Omega$ که پیشامد نامیده می‌شوند. \\
	\textbf{تابع $P(A)$:}
	طبق اصول موضوعه، به هر پیشامد $A$ عددی نسبت می‌دهد.
	
	\subsection*{فضای نمونه‌ها}
	\textbf{آزمایش تصادفی:}
	آزمایشی است که نتیجه آن از پیش مشخص نیست اما در عین حال همه نتایج آن قابل پیش‌بینی هستند. \\
	\textbf{فضای نمونه‌ها:}
	مجموعه کلیه نتایج ممکن برای یک آزمایش تصادفی است و با $\Omega$ و یا $S$ نمایش داده می‌شود.\\ 
	\textbf{پیشامد:}
	هر زیرمجموعه‌ای از فضای نمونه یک پیشامد است. می‌گوییم پیشامد $A$ اتفاق افتاده هرگاه نتیجه آزمایش یکی از اعضای مجموعه $A$ باشد.\\
	\textbf{انواع پیشامد:}\\
	\textbf{پیشامد حتمی $(Sure Event)$:}
	پیشامد $\Omega$ پیشامد حتمی است. \\
	\textbf{پیشامد ناممکن $(Null Event)$:}
	پیشامد $\Omega$ پیشامد ناممکن یا خنثی است. \\
	\textbf{دو پیشامد ناسازگار:}
	$A$ و $B$ دو پیشامد ناسازگارند هرگاه مجموعه‌های $A$ و $B$ جدا از هم باشند.
	\\ 
	\textbf{پیشامد ساده:}
	پیشامدی که تنها یک عضو داشته باشد پیشامد ساده است.

\end{document}